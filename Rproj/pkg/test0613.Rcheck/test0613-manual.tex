\nonstopmode{}
\documentclass[a4paper]{book}
\usepackage[times,inconsolata,hyper]{Rd}
\usepackage{makeidx}
\usepackage[utf8,latin1]{inputenc}
% \usepackage{graphicx} % @USE GRAPHICX@
\makeindex{}
\begin{document}
\chapter*{}
\begin{center}
{\textbf{\huge Package `test0613'}}
\par\bigskip{\large \today}
\end{center}
\begin{description}
\raggedright{}
\item[Type]\AsIs{Package}
\item[Title]\AsIs{What the package does (short line)}
\item[Version]\AsIs{1.0}
\item[Date]\AsIs{2013-06-13}
\item[Author]\AsIs{Who wrote it}
\item[Maintainer]\AsIs{Who to complain to }\email{yourfault@somewhere.net}\AsIs{}
\item[Description]\AsIs{More about what it does (maybe more than one line)}
\item[License]\AsIs{What license is it under?}
\end{description}
\Rdcontents{\R{} topics documented:}
\inputencoding{utf8}
\HeaderA{test0613-package}{test0613 \textasciitilde{}\textasciitilde{} package title \textasciitilde{}\textasciitilde{}}{test0613.Rdash.package}
\aliasA{test0613}{test0613-package}{test0613}
\keyword{package}{test0613-package}
%
\begin{Description}\relax
More about what it does (maybe more than one line)
\textasciitilde{}\textasciitilde{} A concise (1-5 lines) description of the package \textasciitilde{}\textasciitilde{}
\end{Description}
%
\begin{Details}\relax

\Tabular{ll}{
Package: & test0613\\{}
Type: & Package\\{}
Version: & 1.0\\{}
Date: & 2013-06-13\\{}
License: & What license is it under?\\{}
}
\textasciitilde{}\textasciitilde{} An overview of how to use the package, including the most important \textasciitilde{}\textasciitilde{}
\textasciitilde{}\textasciitilde{} functions \textasciitilde{}\textasciitilde{}
\end{Details}
%
\begin{Author}\relax
Who wrote it

Maintainer: Who to complain to <yourfault@somewhere.net>
\textasciitilde{}\textasciitilde{} The author and/or maintainer of the package \textasciitilde{}\textasciitilde{}
\end{Author}
%
\begin{References}\relax
\textasciitilde{}\textasciitilde{} Literature or other references for background information \textasciitilde{}\textasciitilde{}
\end{References}
%
\begin{SeeAlso}\relax
\textasciitilde{}\textasciitilde{} Optional links to other man pages, e.g. \textasciitilde{}\textasciitilde{}
\textasciitilde{}\textasciitilde{} \code{\LinkA{<pkg>}{<pkg>}} \textasciitilde{}\textasciitilde{}
\end{SeeAlso}
\inputencoding{utf8}
\HeaderA{mytestfun}{hello}{mytestfun}
\keyword{\textbackslash{}textasciitilde{}kwd1}{mytestfun}
\keyword{\textbackslash{}textasciitilde{}kwd2}{mytestfun}
%
\begin{Usage}
\begin{verbatim}
mytestfun(a, b)
\end{verbatim}
\end{Usage}
%
\begin{Arguments}
\begin{ldescription}
\item[\code{a}] 


\item[\code{b}] 


\end{ldescription}
\end{Arguments}
%
\begin{Examples}
\begin{ExampleCode}
##---- Should be DIRECTLY executable !! ----
##-- ==>  Define data, use random,
##--	or do  help(data=index)  for the standard data sets.

## The function is currently defined as
function (a, b) 
{
    print(a)
    print(b)
    print(a + b)
    return(a * b)
  }
\end{ExampleCode}
\end{Examples}
\inputencoding{utf8}
\HeaderA{result}{hello\_result}{result}
\keyword{datasets}{result}
%
\begin{Usage}
\begin{verbatim}
data(result)
\end{verbatim}
\end{Usage}
%
\begin{Format}
The format is:
num 200
\end{Format}
%
\begin{Examples}
\begin{ExampleCode}
data(result)
## maybe str(result) ; plot(result) ...
\end{ExampleCode}
\end{Examples}
\printindex{}
\end{document}
